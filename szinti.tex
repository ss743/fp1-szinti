\documentclass[12pt,listof=totoc]{scrartcl}

\usepackage{fancyhdr}
\usepackage{geometry}
\usepackage{ucs}
\usepackage[utf8x]{inputenc}
\usepackage[T1]{fontenc}
\usepackage[ngerman]{babel}
\usepackage{amsmath,amssymb,amstext}
\usepackage{hyperref}
\usepackage{cancel}
\usepackage{dsfont}
\usepackage{physics}
\usepackage{lmodern}
\usepackage{enumerate}
\usepackage{enumitem}
\usepackage{graphicx}
\usepackage{listings, color}
\usepackage[labelfont=bf]{caption}
\usepackage{titling}

\lstset{basicstyle=\scriptsize} %Quellcode mit Umlauten und ganz klein
\lstset{literate=
  {Ö}{{\"O}}1
  {Ä}{{\"A}}1
  {Ü}{{\"U}}1
  {ß}{{\ss}}2
  {ü}{{\"u}}1
  {ä}{{\"a}}1
  {ö}{{\"o}}1
}


%Geometrie----------------------------------------------------------------------------------------------------------

\geometry{a4paper, top=25mm, left=15mm, right=15mm, bottom=25mm,headsep=10mm, footskip=10mm}
\pagestyle{fancy}
\setlength{\parindent}{0pt} %Zeileneinrückung

\fancyhf{} %Setzt voreingestellte Kopf-und Fußzeilen-Eigenschaften zurück

\lhead{\nouppercase{\leftmark}}
\chead{}
\rhead{\thepage}

\lfoot{}
\cfoot{}
\rfoot{}

\title{\vspace{0cm}{\Huge Fortgeschrittenen-Praktikum I:\\ \vspace{1cm} Szintillationszähler}}
\author{Saskia Bondza\\Simon Stephan}
\date{durchgeführt am 12.10.2016 und 13.10.2016}

\pretitle{%
  \begin{center}
  \LARGE
  \includegraphics[width=6cm,]{figures/siegel}\\[\bigskipamount]
}
\posttitle{\end{center}}

%neue Commands----------------------------------------------------------------------------------------------------------
\newcommand{\nab}{\vec{\nabla}} %direkter Befehl mit Vektorpfeil
\newcommand{\gra}[3][0.7]{
	\begin{minipage}[h!]{\textwidth}
		\centering
		\includegraphics[width=#1\textwidth]{figures/#2.png}
		\captionof{figure}{#3}
	\end{minipage}
	\vskip 30 pt
}
\newcommand{\graTwo}[4][0.49]{
	\begin{minipage}[h!]{\textwidth}
		\centering
		\includegraphics[width=#1\textwidth]{figures/#2.png}
		\includegraphics[width=#1\textwidth]{figures/#3.png}
		\captionof{figure}{#4}
	\end{minipage}
	\vskip 30 pt
}
\newcommand{\graTwoB}[5]{
	\begin{minipage}[h!]{\textwidth}
		\centering
		\includegraphics[width=#1\textwidth]{figures/#3.png}
		\includegraphics[width=#2\textwidth]{figures/#4.png}
		\captionof{figure}{#5}
	\end{minipage}
	\vskip 30 pt
}
\newcommand{\graThree}[5][0.49]{
	\begin{minipage}[h!]{\textwidth}
		\centering
		\includegraphics[width=#1\textwidth]{figures/#2.png}
		\includegraphics[width=#1\textwidth]{figures/#3.png}\newline
		\includegraphics[width=#1\textwidth]{figures/#4.png}
		\captionof{figure}{#5}
	\end{minipage}
	\vskip 30 pt
}
\newcommand{\del}[2][]{\frac{\partial #1}{\partial #2}}
\newcommand{\code}[1]{\texttt{#1}}


%Titel,Inhalt----------------------------------------------------------------------------------------------------------

\begin{document}
\pagenumbering{gobble} %verstecke Seitenzahl
\maketitle
\newpage

\section*{Abstract}



\newpage

\thispagestyle{empty}
\tableofcontents
\newpage

%Schreiben----------------------------------------------------------------------------------------------------------
\pagenumbering{arabic} %verstecke Seitenzahl
\section{Einleitung}




\newpage
\section{Theoretische Grundlagen}

\subsection{Zerfallsgesetz}\label{zerfallsgesetz}
Der radioaktive Zerfall ist ein statistischer Prozess, d.h. der genaue Zeitpunkt, zu dem ein Kern zerfällt, kann nicht vorhergesagt werden. Wir können nur Wahrscheinlichkeiten angeben. Die Zerfallsrate $dN/dt$ ist abhängig von der momentanen Anzahl $N$ der Atome.
\[\frac{dN}{dt}=-\lambda N\]
wobei $\lambda$ die Zerfallskonstante ist, welche den Bruchteil der zu Beginn vorhandenen Kerne, die pro Zeiteinheit zerfallen beschreibt. Zerfallskonstanten einzelner Zerfälle addieren sich, zerfällt ein Kern auf mehrere Arten. Durch Lösen dieser Differentialgleichung erhält man die Anzahl der Kerne $N$ zum Zeitpunkt $t$.
\[N(t)=N_0e^{-\lambda t}\hspace{0.5cm}\]
wobei $N_0=N(t=0)$ die anfängliche Anzahl der Kerne beschreibt. Hieraus lässt sich dann die mittlere Lebensdauer $\tau$ bestimmen:
\[\tau=\frac{1}{\lambda}\]
welche mit der Halbwertszeit $T_{1/2}$ wie folgt zusammenhängt:
\[T_{1/2} = \ln2\cdot \tau = \frac{\ln2}{\lambda}\]
%Eine weitere wichtige Größe ist die Aktivität $A$ welche die Anzahl der Zerfälle pro Sekunde beschreibt und in Becquerel ( $Bq=1/s$) gemessen wird:
%\[A = \lambda N = \frac{N}{\tau} = \frac{N\ln2}{T_{1/2}}\]
\subsection{Radioaktive Zerfälle}
Instabile Atomkerne gehen, je nach Art des Zerfalls, unter Aussendung von ionisierender Strahlung und Aussendung von Teilchen spontan in einen anderen Atomkern über. Im Folgenden werden die verschiedenen Arten radioaktiver Zerfälle erläutert.
\subsubsection{$\alpha$-Zerfall}
Beim $\alpha$-Zerfall geht ein schwerer, instabiler Kern unter Aussendung eines $\alpha$-Teilchens, einem Heliumkern (zwei Protonen und zwei Neutronen) in einen stabilen Kern über. Die Massenzahl des neuen Atomkerns ist dabei um vier gesunken, die Ordnungszahl verringert sich um zwei.

\begin{align}
{}_Z^A X \rightarrow {}_{Z-2}^{A-4} Y + {}^4_2He
\end{align}

\subsubsection{$\beta$-Zerfall}
Der $\beta$-Zerfall wird durch verschiedene Prozesse der schwachen Wechselwirkung beschrieben bei denen Elektronen  und Positronen auftreten. Bei diesem Vorgang wird durch W-Boson-Austausch ein Proton in ein Neutron umgewandelt bzw. ein Neutron in ein Proton.

\paragraph*{$\beta^-$-Zerfall:} 
Bei dieser Art des Zerfalls wandelt sich ein Neutron im Kern unter Aussendung eines Elektrons und eines Elektron-Antineutrinos in ein Proton um. 
\begin{align*}
n &\rightarrow p + e^- + \bar{\nu_e}\\
{}_Z^A X &\rightarrow {}_{Z+1}^A Y + e^- + \bar{\nu_e}
\end{align*}

Hierbei entspricht X dem Mutternuklid und Y dem Tochternuklid. Die Ordnungszahl erhöht sich bei diesem Vorgang um eine Einheit, die Massenzahl ändert sich nicht.

\paragraph*{$\beta^+$-Zerfall:}
Der $\beta^+$-Zerfall ist dem $\beta^-$-Zerfall sehr ähnlich, Hier wandelt sich ein Proton im Kern unter Aussendung eines Positrons und eines Elektron-Neutrinos in ein Neutron um.
\begin{align*}
p &\rightarrow n + e^+ + \nu_e\\
{}_Z^A X &\rightarrow {}_{Z-1}^A Y + e^+ + \nu_e
\end{align*}

Bei diesem Prozess sinkt die Kernladungszahl um eine Einheit während die Massenzahl wie auch beim $\beta^-$-Zerfall unverändert bleibt.

\subsubsection{Elektroneneinfang}
Die meisten schweren, instabilen Kerne mit Protonenüberschuss wandeln sich durch Elektroneneinfang in stabile Kerne um.
Der Elektroneneinfang führt effektiv im Kern zum gleichen Resultat wie der $\beta^+$-Zerfall, d.h. die Ordnungszahl verringert sich um eine Einheit, die Massenzahl bleibt unverändert. Meist wird aus der Kern-nächsten Schale, der K-Schale, ein Bahnelektron unter Aussendung eines Neutrinos eingefangen. Die Lücke, die hierbei entsteht, wird meist durch ein aus der L-Schale stammendes Elektron unter Emission von Röntgenstrahlung oder Auger-Elektronen (siehe \ref{auger}) gefüllt. Dieser Vorgang wiederholt sich mit den weiter außen liegenden Schalen.
\[{}_Z^A X + e^- \rightarrow\ {}_{Z-1}^A Y + \nu_e\]
\subsubsection{$\gamma$-Zerfall}

Der $\gamma$-Zerfall ist oft eine Begleiterscheinung der bereits diskutierten Kernzerfälle: Befindet sich das Tochternuklid nach dem Zerfall in einem angeregten Zustand, geht es über Aussendung eines $\gamma$-Quants (Photon)  in einen energetisch niedrigeren Zustand über. Die Energie dieser Photonen liegt dabei im Bereich von keV bis MeV. In Materie fällt die Intensität $I_d$ der $\gamma$-Strahlung exponentiell mit der Strecke $d$ ab:
\[I_d=e^{\mu d}\]
Dabei ist $\mu$ der Absorptionskoeffizent des Materials und $I_0$ die Anfangsintensität.

\subsubsection{Innere Konversion}
Konkurrierend zum $\gamma$-Zerfall gibt es auch die Innere Konversion. Bei dieser gehen angeregte Kernzustände strahlungslos in den Grundzustand über, wobei die Energie an ein Elektron übertragen und dieses abgestrahlt wird. Die so entstandene Lücke in der Atomhülle wird dann unter Emission von Röntgenstrahlung oder Auger-Elektronen (siehe \ref{auger}) gefüllt.

\subsection{Prozesse in der Atomhülle}\label{auger}
Bei manchen Zerfallsprozessen entsteht in der Atomhülle durch Emission eines Elektrons ein Loch. Dieses Loch wird gefüllt, indem ein Elektron einer höheren Schale in die Schale des Loches wechselt und dabei Energie freigibt. Diese Energie kann entweder über direkte Emission eines Photons (Röntgenstrahlung) oder über die Emission eines anderen Elektrons (Auger-Elektron) abgegeben werden. Dadurch entstehen ein oder zwei neue Lücken, welche wiederum über die weitere Emission von Röntgenstrahlung oder eines Auger-Elektrons geschlossen werden können.

\subsection{Wechselwirkung von $\gamma$-Strahlung mit Materie}
Um $\gamma$-Strahlung detektieren zu können, muss die $\gamma$-Strahlung mit Materie wechselwirken. Wir unterscheiden dabei grundsätzlich drei Prozesse: Den Photoeffekt, den Compton-Effekt und die Paarbildung.
 \subsubsection{Photo-Effekt}
 Beim Photoeffekt dringt ein Photon in das Atom ein und überträgt seine gesamte Energie an ein Elektron der inneren Schalen. Dabei wird Energie auf dieses Elektron übertragen, es wird aus der Atomhülle befreit und erhält kinetische Energie. Die hier entstandene Lücke wird über Abstrahlung eines $\gamma$-Quants oder eines Auger-Elektrons wieder gefüllt.
 Der Photoeffekt findet typischerweise bei Energien bis zu $200$ keV statt.
 
 \gra{Photo}{Äußerer Photoeffekt.}
 \subsubsection{Compton-Effekt}
 Beim Compton-Effekt trifft ein einfallendes $\gamma$-Quant auf ein freies oder nur leicht gebundenes Elektron und überträgt einen Teil seiner Energie auf dieses. Dieser Prozess findet meist bei Energien zwischen $200$ keV und $5$ MeV statt.
 
 \gra{Compton}{links: In dieser Darstellung kann man ein Photon sehen, das mittels Comptonef-
 fekt an einem Elektron gestreut wird und dabei Energie verliert, da es nach dem Stoß langwelliger
 ist. rechts: Dieser Graph skizziert den erwateten Verlauf im Energiespektrum einer einzelnen
 Photonenenregie durch Compton- und Photoeffekt. Die Wölbung des Comptonplateaus ist nicht zu
 sehen.}
 \subsubsection{Paarbildung und Paarvernichtung}
 Bei der Paarbildung entsteht durch die Wechselwirkung des $\gamma$-Quants mit dem elektromagnetischen Feld des Atomkerns oder eines Elektrons ein Teilchen-Antiteilchen-Paar, z.B. Elektronen-Positronen-Paar.Paarbildung ist für Energien über $1,022$ MeV möglich. Die über diesen Grenzwert hinausgehende Energie wird auf die entstandenen Teilchen übertragen, der Imuls wird vom Kern aufgenommen. Da das Positron nicht lange alleine existieren kann, vereinigt es sich unter Abstrahlung von zwei $\gamma$-Quanten mit einem Elektron.
 
 \subsection{Zerfallsreihen der verwendeten Präparate \label{ZR}}
 
 Das Produkt eines radioaktiven Zerfalls kann instabil sein und ebenfalls zerfallen. Diese
 Abfolge von Zerfällen wird Zerfallsreihe genannt. Nachfolgend werden die Zerfallsreihen der verwendeten radioaktiven Isotope und die sich hieraus ergebenden erwarteten Peaks im Spektrum erläutert.
 
 \subsubsection{Natrium $^{22}Na$}
 
 \gra{Zerfallsreihe-Na}{Zerfallsreihe von $^{22}Na$}
 
 \subsubsection{Cobalt  $^{60}Co$}
 $^{60}Co$ zerfällt durch zwei $\beta^-$ -Zerfälle mit annähernd 100 \% in den angeregten Zustand 4+ von  $^{60}Ni$, wobei die Halbwertszeit $T=5,2714$ Jahre beträgt. Die höchste Wahrscheinlichkeit hat dann der Übergang vom 4+ in den 2+ Kernzustand. Im Energiespektrum von $^{60}Co$ sind somit Peaks bei $1173,2$ keV und $1332,5$ keV  zu erwarten, wobei ersterer der Energiedifferenz zwischen dem 4+ und dem 2+ Kernzustand entspricht und der zweite der Energiedifferenz des Übergangs in den Grundzustand (siehe Abbildung \ref{Co} )
 
 \gra{Zerfallsreihe-Co}{Zerfallsreihe von $^{60}Co$ \label{Co} }
 
 \subsubsection{Europium  $^{152}Eu$}
 Das Europium Isotop $^{152}Eu$ zerfällt mit einer Halbertszeit von 12 Jahren durch $\beta^-$ -Zerfall in   $^{152}_{64}Gd$ sowie $\beta^+$ -Zerfall und Electron Capture in $^{152}_{62}Sm$. Der Übergang der angeregten Zustände von  $^{152}_{64}Gd$ und $^{152}_{62}Sm$ in den Grundzustand entspricht Peaks bei 122 keV und 344 keV. In Abbildung \ref{Eu} sind die für diesen Versuch relevanten Zerfälle dargestellt.
 
 
 \gra{Zerfallsreihe-Eu}{Vereinfachte Zerfallsreihe von $^{152}Eu$ \label{Eu}}
 
  \subsubsection{Thorium  $^{228}Th$}
  In Abbildung \ref{Th} ist die Zerfallsreihe von Thorium  $^{232}Th$ dargestellt, die $^{228}Th$ als Zerfallsprodukt enthält. Es wird hier auf Grund der Komplexität dieser Zerfallsreihe auf die Darstellung der wahrscheinlichsten Übergänge und deren zugehörigen Peaks verzichtet und auf REF verwiesen.
  
  \gra{Zerfallsreihe-Th}{Zerfallsreihe von  $^{228}Th$ \label{Th}}
 
 
 
 \subsection{Szintillationszähler}
 
 In Abbildung \ref{Szintiyay} ist die Funktionsweise eines Szntillators dargestellt.Mit Szintillatoren bezeichnet man spezielle Messgeräte, mit denen hochenergetische Photonen in Strahlung niedrigerer Energie aber höherer Intensität umgewandelt werden. Die Intensität der Szintillatorstrahlung ist proportional zur Energie der zu detektierenden Strahlung. Im Szintillationsmedium regt
 die radioaktive Strahlung durch Kernprozesse die Emission von Photonen an. Diese Photonen können das Medium durchdringen und werden in einem Photomultiplier verstärkt und gemessen.
 
 \gra{Szinti}{Funktionsweise eines Szintillationszähler \cite{Demtröder} \label{Szintiyay}}
 
 \subsubsection{Organischer Szintillator}
 
 \subsection{Gerätebeschreibung}
 \paragraph{Detektor} Der Szintillator und der Photomultiplier bilden zusammen den Detektor.
 \paragraph{Pre-Amplifier}
 Der Pre-Amplifier (PA) verstärkt die schwachen Signale des Detektors und verbessert so
 das Signal zu Rausch Verhältnis. Um Signalverluste entlang der Kabel zu reduzieren, ist der PA im Gehäuse des PM integriert. Die Pulsform des Ausgangssignals besitzt eine negative Amplitude und nimmt exponentiell zu.
 \paragraph{Main Amplifier (MA)}
 Der Hauptverstärker  verstärkt das Spannungssignal rauscharm und erzeugt einen möglichst kurzen Puls, dessen Dauer Amplituden-unabhängig ist. Die Pulshöhe ist dabei annähernd proportional zur deponierten Ladungsmenge im Szintillator. Der Hauptverstärker hat zwei verschiedene Ausgänge, die entweder ein bipolares Signal (zeitsensible Messungen) oder ein unipolares Signal (Aufnahme der Spektren) liefern.
 \paragraph{Multi Channel Analyser (MCA)}
 Der Multi Channel Analyser ordnet jeden eingehenden Puls, abhängig von der Pulshöhe (also von der im Szintillator deponierten Ladungsmenge ) einem Channel zu. Das so erhaltene Energiespektrum kann als Histogramm dargestellt werden.
 \paragraph{Single Channel Analyser}
 Ein Single Channel Analyser selektiert aus eingehenden Signalen in dem nur für Energien in einem einstellbaren Energiefenster ein Ausgangspuls erzeugt wird.  Für die Messung der verzögerten und zufälligen Koinzedenzen sollen hier die Energiefenster dem $122$ keV Peak und dem $14,4$ keV Peak angepasst werden (siehe \ref{Energiefenster})
 \paragraph{Gate} Das Gate hat zwei Eingänge. Sofern am enable Eingang ein Signal ankommt, wird das Signal des anderen Eingangs während dieser Zeit direkt als Output weitergeleitet. Die Dauer des Gatesignals kann mit einem Schlitzschraubenzieher eingestellt werden.
 \paragraph{Timing Unit}
 In der Timing Unit werden negative logische Signale verarbeitet und daraus ein rechteckiges Ausgangssignal erzeugt, dessen Breite sich mit Hilfe eines Schlitzschraubenziehers einstellen lässt.
 \paragraph{Coincidence Unit}
 Man spricht von einer Koinzidenz wenn zwei oder mehr Signale zeitgleich auftreten. Sofern eine Koinzidenz (dies kann durch Überschreiten einer Schwelle von der Summe beider Signale erfasst werden) vorliegt, erzeugt die Koinzidenz Unit ein logisches Ausgangssignal.
 \paragraph{HEX Counter}
Der HEX-Counter zählt die eintreffenden Signale sowie die Zeit.

\subsection{Energiespektrum}

Das Energiespektrum, das aus den gemessenen Photonenenergien ensteht, weißt  charakteristische Merkmale auf. Diese sind im Folgenden erklärt. 
 
 
 \paragraph{Photopeak}
 
 Wird die gesamte Energie eines von der Quelle stammenden Photons im Szintillator absorbiert, so wird dies im Spektrum einem Photopeak zugeordnet. Die zu erwarteten Photopeaks sind vom Präparat abhängig und sind in Kapitel \ref{ZR} erklärt.
 \paragraph{Escape-Peak}
 Wenn ein Röntgenquant den Szintillator ohne Wechselwirkung verlassen kann, folgt ein Verlust der Differenz der Bindungsenergie zwischen K-, und L-Schale. Diese Differenz beträgt im Fall des NaI-Szintillators $28$ keV. Damit sind Escapepeaks bei $E_\gamma - 28$ keV zu erwarten, wobei $E_\gamma$ die Energie eines Photopeaks ist.

 \paragraph{Compton-Kante}
 Die maximale Energie, die ein Photon durch den Compton-Effekt auf ein Elektron übertragen kann, entspricht im Spektrum der Compton-Kante. Diese ist charakterisiert durch einen starken Abfall der Zählrate.
 
 \paragraph{Rückstreupeak}
 
 Die Rückstreuung unter einem Winkel von 180 ° hat beim Compton-Effekt die höchste Wahrscheinlichkeit.
 Man beobachtet daher einen Rückstreu-Peak im Spektrum unterhalb der Compton-Kante.

 
 \paragraph{Röntgen-Fluoreszenz-Peak}
Es können in der Quelle Wechselwirkungen von $\gamma$-Quanten mit Elektronen stattfinden, ohne dass diese vom Szintillator detektiert werden. Geht ein Elektron von einer höheren Schale in jene Schale über aus der ein Elktron ausgetreten ist, kann ein Auger-Elektron oder charakteristische Röntgenstrahlung freigesetzt werden. Die Energie dieser charakteristischen Röntgenstrahlung ist dann als Röntgen-Fluoreszenz-Peak im Energiespektrum zu beobachten.




\newpage
\section{Versuchsaufbau und -Durchführung}

\subsection{Teil I: Vorbereitung und Energiekalibrierung}

\subsection{Teil II: Messung der Koinzedenzen}




\newpage
\section{Auswertung}Die Auswertung haben wir mit der Programmiersprache \code{R} durchgeführt. Der Quellcode befindet sich im Anhang in Kapitel \ref{sourcecode}.
\subsection{Aufnahme des Thoriumspektrums}
Um das aufgenommene Thoriumspektrum auswerten zu können, müssen wir den Kanälen des MCA die entsprechende Energie zuordnen. Dazu haben wir für Cobalt, Natrium und Europium ebenfalls Spektren aufgenommen und aus den bekannten Peaks für diese Proben die Energiekalibrierung durchgeführt. Um die Spektren zu bereinigen, haben wir eine Untergrundmessung durchgeführt und die Zählraten des Untergrunds von den gemessenen Zählraten abgezogen, um äußere Einflüsse auf die Spektren zu verringern (siehe \ref{untergrund}). Der Fehler auf die Zählraten ergibt sich durch den Poissonfehler $s_N=\sqrt{N}$ auf die Ereigniszahlen und ist damit $s_n=\frac{\sqrt{N}}{t}=\sqrt{\frac{n}{t}}$.
\newpage
\subsubsection{Energiekalibrierung}
\paragraph{Cobaltspektrum}\ \\
\gra[0.9]{cobalt}{Cobaltspektrum zur Bestimmung der Energiekalibrierung\label{cobaltbild}}
Zunächst bestimmen wir die Positionen der Peaks im Cobaltspektrum und ordnen diese den bekannten Energien des Cobaltspektrums zu. Die Positionen der Peaks erhalten wir durch Fitten einer Gaußfunktion an die Zählraten: $$f(x)=C+N\cdot\exp(-\frac{(x-\mu)^2}{2\sigma^2})$$ Die dabei bestimmten Fitparameter sind in Tabelle \ref{cobalttable} im Anhang zu finden. Aus dem Fit erhalten wir folgende Ergebnisse:
\begin{table}[h!]
	\centering
	\begin{tabular}{c|c|c|c}
		&Position (Kanal)&$\frac{\chi^2}{\text{ndf}}$& Energie / keV (nach \cite{staat})\\\hline
		\textbf{Peak 1}&$\mathbf{2640\pm30}$&$\mathbf{1.9}$&$\mathbf{1173.2}$\\
		\textbf{Peak 2}&$\mathbf{2980\pm40}$&$\mathbf{2.4}$&$\mathbf{1332.5}$\\
		Peak 3&$5630\pm60$&$1.5$&$-$\\		
	\end{tabular}
	\caption{Kanäle der gefundenen Cobaltpeaks\label{cobaltpeaks}}
\end{table}

Die Fehler auf die Positionen der Peaks ergeben sich dabei aus dem Fitparameter $\sigma$ mit $s=\frac{\sigma}{2}$.
In Abbildung \ref{cobaltbild} ist das Cobaltspektrum mit den Gaußfits zu sehen. Dabei sind die bekannten Peaks rot und die unbekannten Peaks rosa eingezeichnet. 
\newpage
\paragraph{Natriumspektrum}\ \\
\gra[0.9]{natrium}{Natriumspektrum zur Bestimmung der Energiekalibrierung\label{natriumbild}}
Nun bestimmen wir die Positionen der Peaks im Natriumspektrum und ordnen diese den bekannten Energien der Natriumpeaks zu. Die Positionen der Peaks erhalten wir durch Fitten einer Gaußfunktion an die Zählraten: $$f(x)=C+N\cdot\exp(-\frac{(x-\mu)^2}{2\sigma^2})$$ Die dabei bestimmten Fitparameter sind in Tabelle \ref{natriumtable} im Anhang zu finden. Aus dem Fit erhalten wir folgende Ergebnisse:

\begin{table}[h!]
	\centering
	\begin{tabular}{c|c|c|c}
		&Position (Kanal)&$\frac{\chi^2}{\text{ndf}}$& Energie / keV (nach \cite{staat})\\\hline
		\textbf{Peak 1}&$\mathbf{1170\pm50}$&$\mathbf{17}$&$\mathbf{511}$\\
		\textbf{Peak 2}&$\mathbf{2830\pm50}$&$\mathbf{7.5}$&$\mathbf{1274.6}$\\
		Peak 3&$3980\pm120$&$8.8$&$-$\\		
	\end{tabular}
	\caption{Kanäle der gefundenen Natriumpeaks\label{natriumpeaks}}
\end{table}

Die Fehler auf die Positionen der Peaks ergeben sich dabei aus dem Fitparameter $\sigma$ mit $s=\frac{\sigma}{2}$.
In Abbildung \ref{natriumbild} ist das Natriumspektrum mit den Gaußfits zu sehen. Dabei sind die bekannten Peaks rot und die unbekannten Peaks rosa eingezeichnet.
\newpage
\paragraph{Europiumspektrum}\ \\
\gra[0.9]{europium}{Europiumspektrum zur Bestimmung der Energiekalibrierung}
Nun bestimmen wir die Positionen der Peaks im Europiumspektrum und ordnen diese den bekannten Energien der Europiumpeaks zu. Die Positionen der Peaks erhalten wir durch Fitten einer Gaußfunktion an die Zählraten: $$f(x)=C+N\cdot\exp(-\frac{(x-\mu)^2}{2\sigma^2})$$ Die dabei bestimmten Fitparameter sind in Tabelle \ref{europiumtable} im Anhang zu finden. Aus dem Fit erhalten wir folgende Ergebnisse:

\begin{table}[h!]
	\centering
	\begin{tabular}{c|c|c|c}
		&Position (Kanal)&$\frac{\chi^2}{\text{ndf}}$& Energie / keV (nach \cite{staat})\\\hline
		Peak 1&$105\pm6$&$25$&$-$\\
		Peak 2&$199\pm9$&$16$&$-$\\
		\textbf{Peak 3}&$\mathbf{302\pm7}$&$\mathbf{4.2}$&$\mathbf{122}$\\
		Peak 4&$583\pm11$&$1.3$&$245$\\
		\textbf{Peak 5}&$\mathbf{810\pm15}$&$\mathbf{1.8}$&$\mathbf{344}$\\
		Peak 6&$1780\pm20$&$1.1$&$779$\\
		Peak 7&$2190\pm30$&$1.1$&$964$\\
		Peak 8&$2480\pm40$&$1.3$&$1086\text{ bzw. }1112$\\		
		Peak 9&$3160\pm30$&$1.1$&$1408$\\		
	\end{tabular}
	\caption{Kanäle der gefundenen Europiumpeaks\label{europiumpeaks}}
\end{table}

Die Fehler auf die Positionen der Peaks ergeben sich dabei aus dem Fitparameter $\sigma$ mit $s=\frac{\sigma}{2}$.
In Abbildung \ref{natriumbild} ist das Spektrum der Europiumprobe mit den Gaußfits zu sehen. Dabei sind die bekannten Peaks rot und die unbekannten Peaks rosa eingezeichnet. Zur Energiekalibrierung verwenden wir von Europium die beiden bekannten Peaks der höchsten Intensitäten.
\newpage
\paragraph{Energiekalibrierung}\ \\
\gra{calibration}{Energiekalibrierung mithilfe der bekannten Peaks von Cobalt, Natrium und Europium}

Anschließend tragen wir die Kanäle der gemessenen Peaks über die dazugehörigen Energien auf und fitten daran eine lineare Funktion, um die Umrechnung zwischen Kanälen und Energien zu bestimmen. Aus dem Fit erhalten wir folgende Funktion:
\begin{align*}
\text{Kanal}&=\left(2.217\pm0.012\right)\mathrm{\frac1{eV}}\cdot E + \left(34\pm 5\right)&\text{ mit } \chi^2&=0.34
\end{align*}
Also gilt für die Umrechnung von Kanälen in Energie nun:
\begin{align}
E&=\left(0.451\pm0.002\right)\mathrm{keV}\cdot\text{Kanal}+\left(15\pm2\right)\mathrm{keV}\label{umrechnung}
\end{align}
mit dem Fehler auf die Energie durch Gauß'sche Fehlerfortpflanzung:
\begin{align*}
s_E&=\cdot\sqrt{E^2\cdot\left(\left(\frac{s_{\text{Kanal}}}{\text{Kanal}}\right)^2+\left(\frac{0.002}{0.451}\right)^2\right)+4}
\end{align*}
Im Folgenden werden die Spektren immer mithilfe von (\ref{umrechnung}) über die Energie aufgetragen.

\subsubsection{Untergrund}\label{untergrund}
%\gra[1]{untergrund-0}{Untergrundspektrum}
\gra[1]{untergrund}{Untergrundspektrum\label{background}}
In Abbildung \ref{background} ist das gemessene Untergrundspektrum aufgetragen. In diesem lassen sich zwei Peaks identifizieren, an welche jeweils eine Gaußfunktion gefittet und eingezeichnet wurde: $$f(x)=C+N\cdot\exp(-\frac{(x-\mu)^2}{2\sigma^2})$$ Die dabei bestimmten Fitparameter sind in Tabelle \ref{untergrundtable} im Anhang zu finden. Aus den Fits erhalten wir folgende Energien für die Peaks:

\begin{table}[h!]
	\centering
	\begin{tabular}{c|c|c}
		&Energie / keV&$\frac{\chi^2}{\text{ndf}}$\\\hline
		Peak 1&$1490\pm20$&$1.3$\\
		Peak 2&$2620\pm20$&$0.99$
	\end{tabular}
	\caption{Energien der gefundenen Untergrundpeaks\label{untergrundpeaks}}
\end{table}
\newpage
\subsubsection{Bestimmung der Thorium-Peaks}
Nachdem die Energiekalibrierung durchgeführt und die Untergrundmessung abgezogen wurde, untersuchen wir das Thoriumspektrum und versuchen die Peaks des Thoriumspektrums zu identifizieren. Wir fitten an die Peaks wieder Gaußfunktionen mit: $$f(x)=C+N\cdot\exp(-\frac{(x-\mu)^2}{2\sigma^2})$$ Die dabei bestimmten Fitparameter sind in Tabelle \ref{thoriumtable} im Anhang zu finden.
In Abbildung \ref{thoriumGes} ist nun das gesamte Thoriumspektrum mit allen beobachteten Peaks zu sehen.

\gra[1]{thorium-gesamt}{Gesamtes Thoriumspektrum\label{thoriumGes}}
 \clearpage
Die Fits ergeben folgende Werte für die Energien der Peaks:

 \begin{table}[h!]
 	\centering
 	\begin{tabular}{c|c|c}
 		&Energie / keV&$\frac{\chi^2}{\text{ndf}}$\\\hline
 		Peak 1&$70\pm20$&$1.3$\\
 		Peak 2&$110\pm4$&$0.99$\\
 		Peak 3&$182\pm 5$&$1.3$\\
 		Peak 4&$272\pm 6$&$0.99$\\
 		Peak 5&$301\pm 6$&$1.3$\\
 		Peak 6&$379\pm 13$&$0.99$\\
 		Peak 7&$442\pm 8$&$1.3$\\
 		Peak 8&$551\pm 6$&$0.99$\\
 		Peak 9&$622\pm 11$&$1.3$\\
 		Peak 10&$753\pm 15$&$0.99$\\
 		Peak 11&$870\pm 11$&$1.3$\\
 		Peak 12&$1610\pm 40$&$0.99$\\
 		Peak 13&$2180\pm 100$&$1.3$\\
 		Peak 14&$2640\pm 20$&$0.99$\\
 	\end{tabular}
 	\caption{Energien der gefundenen Thoriumpeaks\label{thoriumpeaks}}

 \end{table}
 

\newpage
\paragraph{Identifizierung der Peaks}\ \\

Um die Peaks zu identifizieren, untersuchen wir drei Bereiche des Thoriumspektrums getrennt (Abbildungen \ref{thorium1}-\ref{thorium3}), da aufgrund der stark unterschiedlichen Peakintensitäten und -Breiten einige Peaks ansonsten nicht gut zu erkennen wären. 

\gra{thorium-1}{Ausschnitt aus dem Thoriumspektrum (Peaks 1 bis 7)\label{thorium1}}
Im ersten Ausschnitt (siehe Abbildung \ref{thorium1}) befinden sich 7 Peaks. Von diesen lassen sich die meisten im Rahmen ihrer Fehler mit Literaturwerten identifizieren, für Peak 2 und 6 fällt dies jedoch schwer. Die Werte der Peaks finden sich in Tabelle \ref{thoriumpeaks1}.
 \begin{table}[h!]
 	\centering
 	\begin{tabular}{c|c|c|c}
 		&Energie / keV&$\frac{\chi^2}{\text{ndf}}$&Literaturwert\textsuperscript{\cite{staat}\cite{energien1}}\\\hline
 		Peak 1&$70\pm20$	&$1.3$&	$84.37$\\
 		Peak 2&$110\pm4$	&$0.99$&	$131.61$\\
 		Peak 3&$182\pm 5$	&$1.3$&	$182.3$\\
 		Peak 4&$272\pm 6$	&$0.99$&	$277.36$\\
 		Peak 5&$301\pm 6$	&$1.3$&	$300.09$\\
 		Peak 6&$379\pm 13$	&$0.99$&	$?$\\
 		Peak 7&$442\pm 8$	&$1.3$&	$452.98$
 	\end{tabular}
 	 \caption{Energien der Thoriumpeaks 1 bis 7 aus Abbildung \ref{thorium1}\label{thoriumpeaks1}}
 \end{table}
 
 Zu Peak 2 lässt sich ein Literaturpeak finden, allerdings ist die gemessene Energie um ca. $6\sigma$ größer als der literaturpeak. Die Verschiebung zu einer höheren Energie, wie sie bei Peak 2 auftritt, könnte zum Beispiel durch Comptoneffekte der nachfolgenden Peaks entstanden sein.
 Im Bereich der Energie von Peak 6 ($379\pm14 \mathrm{keV}$) befindet sich kein erwarteter Peak. Die Herkuinft dieses Peaks ist uns also unbekannt.
 
\gra{thorium-2}{Ausschnitt aus dem Thoriumspektrum (Peaks 8 bis 11)\label{thorium2}}
Im zweiten Ausschnitt (siehe Abbildung \ref{thorium1}) befinden sich 7 Peaks. Von diesen lassen sich die meisten im Rahmen ihrer Fehler mit Literaturwerten identifizieren, für Peak 2 und 6 fällt dies jedoch schwer. Die Werte der Peaks finden sich in Tabelle \ref{thoriumpeaks1}.
\begin{table}[h!]
	\centering
	\begin{tabular}{c|c|c|c}
		&Energie / keV&$\frac{\chi^2}{\text{ndf}}$&Literaturwert\textsuperscript{\cite{staat}\cite{energien1}}\\\hline
		Peak 1&$70\pm20$	&$1.3$&	$84.37$\\
		Peak 2&$110\pm4$	&$0.99$&	$131.61$\\
		Peak 3&$182\pm 5$	&$1.3$&	$182.3$\\
		Peak 4&$272\pm 6$	&$0.99$&	$277.36$\\
		Peak 5&$301\pm 6$	&$1.3$&	$300.09$\\
		Peak 6&$379\pm 13$	&$0.99$&	$?$\\
		Peak 7&$442\pm 8$	&$1.3$&	$452.98$
	\end{tabular}
	\caption{Energien der Thoriumpeaks 1 bis 7 aus Abbildung \ref{thorium1}\label{thoriumpeaks1}}
\end{table}

\gra{thorium-3}{Ausschnitt aus dem Thoriumspektrum (Peaks 12 bis 14)\label{thorium3}}


\newpage
\subsection{Winkelabhängigkeit der Paarvernichtung}

Zur Überprüfung der erwarteten Winkelabhängigkeit der Paarvernichtung, tragen wir die Anzahl der gemessenen Koinzedenzen über den Winkel auf. Wie in REF erklärt, erwarten wir ein Maximum bei 180°, was in diesem Versuchsaufbau einem Winkel von 0° entspricht. Als y-Fehlerbalken sind hier die Poissonfehler $s_N = \sqrt{N}$ aufgetragen, die x-Fehlerbalken entsprehen den abgeschätzten Unsicherheiten auf die Winkel. Hierbei haben wir für alle Winkel die mit einer Einrastfunktion eingestellt werden konnten einen Fehler von $s_\alpha = 0.1°$ abgeschätzt und einen Fehler von 0.5° auf die Winkel für die keine Kerbe zum Einrasten vorhanden war. Die hieraus erhaltene Abbildung ist in \ref{PVN} dargestellt. 

\gra{winkel}{Winkelabhängigkeit der Paarvernichtung \label{PVN}}

An die Werte fitten wir nun eine Gaußfuntkion und erhalten für die einzelnen Parameter folgende Werte:

\begin{align*}
\mu &= (1.50 \pm 0.17) °\\
\sigma &= (4.27 \pm 0.14) °\\
N &= (0.0245 \pm 0.0011) °\\
C &= (0.00107 \pm 0.00013)° \\
\end{align*}

\newpage
\section{Zusammenfassung und Diskussion \label{Diskussion}}


\newpage
\section{Anhang}
\subsection{Fitdaten}

\begin{table}[h!]
	\footnotesize\centering
	\begin{tabular}{|c||c|c|c|c||c|}
		\hline
		Energie / Kanal&$N/\mathrm{s^{-1}}$&$C/\mathrm{s^{-1}}$&$\mu/\mathrm{Kanal}$&$\sigma/\mathrm{Kanal}$&$\chi^2$ / ndf\\\hline\hline2640&$2.209\pm0.011$&$0.686\pm0.011$&$2636.98\pm0.16$&$56.5\pm0.4$&$1.94$\\
		2980&$2.218\pm0.006$&$0.136\pm0.003$&$2984.74\pm0.16$&$70.5\pm0.2$&$2.43$\\
		5630&$0.0497\pm5e-04$&$0.0015\pm2e-04$&$5633\pm1$&$112\pm1.2$&$1.49$\\
		\hline\end{tabular}
	\caption{Fitdaten der Cobaltpeaks\label{cobalttable}}
\end{table}


\begin{table}[h!]
	\footnotesize\centering
	\begin{tabular}{|c||c|c|c|c||c|}
		\hline
		Energie / Kanal&$N/\mathrm{s^{-1}}$&$C/\mathrm{s^{-1}}$&$\mu/\mathrm{Kanal}$&$\sigma/\mathrm{Kanal}$&$\chi^2$ / ndf\\\hline\hline1170&$26.8\pm0.7$&$-3.2\pm0.7$&$1168.7\pm0.3$&$92\pm2$&$17.28$\\
		2830&$3.19\pm0.02$&$1.42\pm0.02$&$2828.2\pm0.5$&$109\pm1$&$7.53$\\
		3980&$3\pm0.6$&$-1.3\pm0.6$&$3984.6\pm0.8$&$230\pm30$&$8.83$\\
		\hline\end{tabular}
	\caption{Fitdaten der Natriumpeaks\label{natriumtable}}
\end{table}


\begin{table}[h!]
	\footnotesize\centering
	\begin{tabular}{|c||c|c|c|c||c|}
		\hline
		Energie / Kanal&$N/\mathrm{s^{-1}}$&$C/\mathrm{s^{-1}}$&$\mu/\mathrm{Kanal}$&$\sigma/\mathrm{Kanal}$&$\chi^2$ / ndf\\\hline\hline105&$5.35\pm0.08$&$0.93\pm0.03$&$104.56\pm0.15$&$11.38\pm0.17$&$25.54$\\
		199&$1.01\pm0.03$&$0.6\pm0.03$&$198.7\pm0.5$&$18.3\pm0.9$&$16.14$\\
		302&$2.82\pm0.02$&$0.606\pm0.009$&$301.74\pm0.1$&$13.37\pm0.11$&$4.15$\\
		583&$0.526\pm0.006$&$0.641\pm0.003$&$582.5\pm0.3$&$-21.9\pm0.3$&$1.32$\\
		810&$1.677\pm0.008$&$0.648\pm0.004$&$810.45\pm0.13$&$30.06\pm0.17$&$1.76$\\
		1780&$0.512\pm0.006$&$0.37\pm0.006$&$1778.5\pm0.3$&$49.2\pm0.7$&$1.07$\\
		2190&$0.441\pm0.009$&$0.249\pm0.009$&$2186.1\pm0.3$&$55.3\pm1.2$&$1.1$\\
		2480&$0.687\pm0.005$&$0.13\pm0.005$&$2484.4\pm0.3$&$71.4\pm0.6$&$1.3$\\
		3160&$0.413\pm0.003$&$0.069\pm0.003$&$3160.8\pm0.3$&$64.1\pm0.6$&$1.15$\\
		\hline\end{tabular}
	\caption{Fitdaten der Europiumpeaks\label{europiumtable}}
\end{table}


\begin{table}[h!]
	\footnotesize\centering
	\begin{tabular}{|c||c|c|c|c||c|}
		\hline
		Energie / keV&$N/\mathrm{s^{-1}}$&$C/\mathrm{s^{-1}}$&$\mu/\mathrm{keV}$&$\sigma/\mathrm{keV}$&$\chi^2$ / ndf\\\hline\hline70&$1\pm3$&$-1\pm3$&$74.6\pm0.2$&$40\pm90$&$1.04$\\
		110&$0.791\pm0.013$&$0.157\pm0.013$&$109.6\pm0.08$&$8.51\pm0.19$&$22.84$\\
		182&$0.208\pm0.003$&$0.155\pm0.003$&$181.85\pm0.06$&$9.2\pm0.18$&$1.49$\\
		272&$0.429\pm0.014$&$0.143\pm0.015$&$271.79\pm0.16$&$12.5\pm0.4$&$2.83$\\
		301&$0.615\pm0.003$&$0.232\pm0.002$&$301.49\pm0.06$&$12.67\pm0.09$&$6.07$\\
		379&$0.62\pm0.05$&$-0.12\pm0.05$&$378.79\pm0.1$&$25.8\pm1.5$&$9.77$\\
		442&$0.2118\pm0.0013$&$0.0843\pm0.0014$&$441.84\pm0.06$&$15.93\pm0.13$&$1.81$\\
		551&$0.0056\pm6e-04$&$0.0253\pm6e-04$&$550.7\pm0.8$&$11.1\pm1.6$&$1.28$\\
		622&$0.0218\pm8e-04$&$0.0143\pm9e-04$&$621.7\pm0.3$&$21.7\pm0.9$&$1.21$\\
		753&$0.0093\pm0.0013$&$0.0101\pm0.0014$&$752.8\pm1.1$&$29\pm4$&$1.16$\\
		870&$0.0448\pm6e-04$&$0.0143\pm7e-04$&$870.26\pm0.11$&$21.1\pm0.4$&$1.32$\\
		1610&$0.00111\pm7e-05$&$0.00075\pm6e-05$&$1607\pm4$&$-71\pm7$&$1.37$\\
		2180&$0.00161\pm0.00015$&$4e-05\pm0.00016$&$2176\pm2$&$203\pm17$&$1.46$\\
		2640&$0.00375\pm4e-05$&$0.00043\pm3e-05$&$2637\pm0.4$&$42.7\pm0.7$&$1.03$\\
		\hline\end{tabular}
	\caption{Fitdaten der Thoriumpeaks\label{thoriumtable}}
\end{table}


\begin{table}[h!]
	\footnotesize\centering
	\begin{tabular}{|c||c|c|c|c||c|}
		\hline
		Energie / keV&$N/\mathrm{s^{-1}}$&$C/\mathrm{s^{-1}}$&$\mu/\mathrm{keV}$&$\sigma/\mathrm{keV}$&$\chi^2$ / ndf\\\hline\hline1490&$0.01089\pm0.00013$&$0.00192\pm1e-04$&$1487.8\pm0.4$&$39.1\pm0.6$&$1.28$\\
		2620&$0.00089\pm3e-05$&$0.00022\pm3e-05$&$2623.9\pm1.3$&$43\pm2$&$0.99$\\
		\hline\end{tabular}
	\caption{Fitdaten der Untergrundpeaks\label{untergrundtable}}
\end{table}

%\subsection{Grafiken}


%\subsection{Tabellen}

%\subsubsection{$\alpha$-Plateau Samarium}
%\lstinputlisting[language=MATLAB]{Rohdaten/alphaPlateau_Sm.txt}


%\newpage
\subsection{Quellcode}
\label{sourcecode}
%\lstinputlisting[language=MATLAB]{Rohdaten/alpha.m}

%\newpage
\subsection{Laborheft}
%\begin{minipage}{\textwidth}
%\centering
%\includegraphics[width=0.9\textwidth]{figures/IMG_20151002_141014.jpg}
%\end{minipage}
\label{Laborbuch}
\newpage
\listoffigures

%Literatur----------------------------------------------------------------------------------------------------------

%\cite{les}
\newpage
\thispagestyle{empty}
\begin{thebibliography}{9}

\bibitem{anleitung}
	M. Köhli,
	\emph{Versuchsanleitung: Szintillationszähler},
	Institut für Mathematik und Physik,
	Albert-Ludwigs-universität,
	2011
\bibitem{staat}
	Tobijas Kotyk,
	\emph{Versuche zur Radioaktivität im Physikalischen Fortgeschrittenen Praktikum an der Albert-Ludwigs-Universität Freiburg},
	Institut für Mathematik und Physik,
	Albert-Ludwigs-universität,
	2011
\bibitem{med}
	Dirk Hünninger, Kieran Maher, uvm.,
	\emph{Physikalische Grundlagen der Nuklearmedizin},
	Wikibooks.org,
	2012
\bibitem{photo}
	\emph{Praktikum im DESY Zeuthen}
	https://www-zeuthen.desy.de/exps/physik\_begreifen/chris/\\Photomultiplier.html,
	Stand: 27.09.15
	
\bibitem{Demtröder}
Wolfgang Demtröder
 \emph{Experimentalphysik 4: Kern-, Teilchen-, und Astrophysik},
 Springer-Verlag,
 4. Auflage,
 2014
 
\bibitem{energien1}
\emph{http://hacol13.physik.uni-freiburg.de/fp/Versuche/FP1/FP1-3-Szintillationszaehler/Anhang/Th-228\_tables.pdf}, Stand: 20.10.2016
 
\end{thebibliography}

\end{document}